\chapter{Memoria descriptiva}\label{cap:memoria}
La problemática que se aborda en el presente trabajo es disminuir el esfuerzo necesario para poner en producción un proyecto de \textbf{Internet of Things (IoT)}.

En concreto, se busca aplicar diversas prácticas de la cultura \textbf{DevOps} sobre el entorno de trabajo IoT, lo cual no está muy difundido actualmente, y representa una mejora en los tiempos de desarrollo, de testeo y de despliegue de las aplicaciones sobre los dispositivos que las corren.

El término \textbf{Pipeline} hace referencia al conjunto de etapas que llevan a cabo en el proceso de desarrollo, y que emplean estas prácticas continuas, que se detallan en el presente informe.

El proyecto busca brindar una solución generalizada para emplearse en diversos proyectos IoT, que puedan ejecutarse en cualquier servidor gracias a la tecnología de los contenedores, y que resulte en una solución para otros desarrolladores del rubro.

Los componentes utilizados son:
\begin{itemize}
    \item Un dispositivo IoT sencillo, como el ESP32
    \item Un servidor para ejecutar las tareas del pipeline, de preferencia Linux
    \item Un cable microUSB para conectar el dispositivo con el servidor
    \item Una placa de pruebas estilo protoboard para probar cómodamente
\end{itemize}

Suponiendo que se usa una computadora propia como servidor, por ejemplo una notebook con Linux, el costo aproximado de los demás componentes es de unos 10 dólares.

